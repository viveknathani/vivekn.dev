%%%%%%%%%%%%%%%%%%%%%%%%%%%%%%%%%%%%%%%%%
% Medium Length Professional CV
% LaTeX Template
% Version 2.0 (8/5/13)
%
% This template has been downloaded from:
% http://www.LaTeXTemplates.com
%
% Original author:
% Trey Hunner (http://www.treyhunner.com/)
%
% Important note:
% This template requires the resume.cls file to be in the same directory as the
% .tex file. The resume.cls file provides the resume style used for structuring the
% document.
%
%%%%%%%%%%%%%%%%%%%%%%%%%%%%%%%%%%%%%%%%%

%----------------------------------------------------------------------------------------
%   PACKAGES AND OTHER DOCUMENT CONFIGURATIONS
%----------------------------------------------------------------------------------------

\documentclass{resume} % Use the custom resume.cls style
\usepackage[left=0.7in,top=0.4in,right=0.7in,bottom=0.5in]{geometry} % Document margins
\usepackage{hyperref}
\usepackage{graphicx}

%----------------------------------------------------------------------------------------
%   DOCUMENT START
%----------------------------------------------------------------------------------------

\name{Vivek Nathani}
\usepackage{graphicx}
\address{
+91 9106581560 \\
viveknathani2402@gmail.com \\
\href{https://GitHub.com/viveknathani/}{\underline{GitHub}} \\
\href{https://www.LinkedIn.com/in/viveknathani/}{\underline{LinkedIn}}
}

\begin{document}
\title{Vivek's Resume}
%----------------------------------------------------------------------------------------
%   EDUCATION SECTION
%----------------------------------------------------------------------------------------

\begin{rSection}{Education}

\begin{rSubsection}{Symbiosis Institute Of Technology}{2019-2023}{B.Tech in Information Technology}{Pune, India}

\vspace{-.65cm}
\item[]
\end{rSubsection}
\end{rSection}


%----------------------------------------------------------------------------------------
%   Projects
%----------------------------------------------------------------------------------------

\begin{rSection}{Experience}
\begin{rSubsection}{Founding Engineer}{August 2024 - Present}
{Mars Computers}{}
\item Working on making powerful compute accessible everywhere.
\item Using Rust and Go.

\end{rSubsection}
\begin{rSubsection}{Software Engineer}{July 2023 - April 2024}{Investmint}{}
\item Built the entire backend for having an in-app broking experience on top of an existing broker. Integrated their APIs into our ecosystem and securely exposed the data for our product. 12000+ orders were processed through this layer inside our app.
\item Integrated Razorpay for putting the in-app broking layer behind a paywall.
\item Built the news feed that sends the latest updates for all the companies who have their stocks listed on NSE. 70000+ news items were processed using OpenAI’s models and 8000+ orders were placed because of these news items.
\item Built the portfolio connecting feature that lets you import your portfolio from 10+ brokers in India and get detailed analysis on top of it. 16000+ portfolios were connected via this layer.
\item Built inves-broker - A heavily used internal library written in Node.js for interacting with multiple broker APIs in the Indian stock market. The launch of this project led to a 96\% increase in in-app orders.
\item Technologies used: Node.js, PostgreSQL, Redis, Docker, MongoDB, and AWS.
\end{rSubsection}
\begin{rSubsection}{Software Engineer Intern}{June - July 2022, December 2022 - June 2023}{Investmint}{}
\item Implemented the REST APIs for displaying data of Indian stocks inside the app. Asynchronous jobs were implemented to process data of 2000+ stocks every day.
\item Was the single person of contact for experimenting with API based trading in the early days via Zerodha and Dhan which eventually led to the inves-broker project.
\item Used technologies like Node.js, PostgreSQL, Redis, and Render.
\end{rSubsection}
\end{rSection}

\begin{rSection}{Projects}

\begin{rSubsection}{teachyourselfmath - A math PDF extraction engine.}{\href{https://teachyourselfmath.app/}{\underline{Live website}}}{Using Node.js, TypeScript, PostgreSQL,Redis}{}
\item Implemented an engine that takes LaTeX outputs of a PDF from an AI model, extracts the relevant math problems, and stores them in a database.
\item Posted about this on Hacker News ({\href{https://news.ycombinator.com/item?id=39113879}{\underline{link}}) - This led to 300+ upvotes and made it to the front page in the top 10 list.
\end{rSubsection}
\end{rSection}

%----------------------------------------------------------------------------------------
%   Skills
%----------------------------------------------------------------------------------------
\begin{rSection}{Skills}

\begin{tabular}{ @{} >{\bfseries}l @{\hspace{6ex}} l }
Languages & C++, JavaScript (primarily Node.js), Go \\
Tools & Git, Linux, GitHub Actions, MySQL, PostgreSQL, MongoDB, AWS \\
\end{tabular}

\end{rSection}

\end{document}
